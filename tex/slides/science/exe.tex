\subsubsection{A wrapper for \ab{} software executables}

\begin{frame}[fragile, allowframebreaks]{\subsubsecname}
    To run external \ab{} software within Julia, what should we do?\\

    Use the Julia \texttt{AbstractCmd}s? Good choice!
    But sometimes, we want to generate a series of \texttt{AbstractCmd}s dynamically,
    especially in a workflow system.
    Writing them one by one is not so efficient.\\

    Wait, can we write functions to generate \texttt{AbstractCmd}s? Then all the command
    arguments will be standard Julia function arguments! For example,

    {\scriptsize
            \begin{algorithmblock}
                \begin{juliaverbatim}
mpirun -np 16 pw.x -npool 2 -nk 2 -input scf_in > scf_out
                \end{juliaverbatim}
            \end{algorithmblock}
        }

    will be

        {\scriptsize
            \begin{algorithmblock}
                \begin{juliaverbatim}
pw("scf_in", "scf_out"; np = 16, npool = 2, nk = 2)
                \end{juliaverbatim}
            \end{algorithmblock}
        }

    \framebreak

    What's more, we could use
    \href{https://github.com/comonicon/Comonicon.jl}{\texttt{Comonicon.jl}},
    to build new executables that will provide customized arguments
    (see \href{https://github.com/MineralsCloud/QuantumESPRESSOCommands.jl}{\texttt{QuantumESPRESSOCommands.jl}})

    {\scriptsize
            \begin{algorithmblock}
                \begin{juliaverbatim}
using Comonicon: @cast
@cast function pw(input, output = mktemp(parentdir(input))[1];
                  np = 1, npool = 1, nk = 1, path = "pw.x", chdir = false)
    ...
    return run(cmd)
end
                \end{juliaverbatim}
            \end{algorithmblock}
        }

    Therefore, we could run it with

        {\scriptsize
            \begin{algorithmblock}
                \begin{juliaverbatim}
pw -np 16 -npool 2 -nk 2 -input scf_in scf_out
            \end{juliaverbatim}
            \end{algorithmblock}
        }

\end{frame}
