\subsubsection{Some simple calculations}

\begin{frame}
    \frametitle{}

    In \ab{} workflows, most of the calculations are done by \ab{} software.
    However, sometimes we do need to do some operations by ourselves when preparing the input
    or analyzing the output, such as unit conversion, fitting, root-finding, and some more
    complicated computations.\\

    In \express{}, \texttt{EquationsOfStateOfSolids.jl} and \texttt{Crystallography.jl}
    are two good examples.\\



\end{frame}

\begingroup
\scriptsize % Change this page's font size, see https://tex.stackexchange.com/a/213839/61591
\begin{frame}[fragile]
    \frametitle{\subsubsecname}

    A 3rd-order Birch--Murnaghan EOS needs 4 parameters: $V_0$, $B_0$, $B_0'$, and optionally
    $E_0$. From them you can calculate $E(V)$, $B(V)$, and $P(V)$ of a solid material.

        {\tiny
            \begin{algorithmblock}
                \begin{juliaverbatim}
struct BirchMurnaghan3rd{T}
    v0::T
    b0::T
    b′0::T
    e0::T
    BirchMurnaghan3rd{T}(v0, b0, b′0, e0 = zero(v0 * b0)) where {T} = new(v0, b0, b′0, e0)
end
        \end{juliaverbatim}
            \end{algorithmblock}
        }

    How to distinguish different equations ($E(V)$, $B(V)$, $P(V)$)? Functions! Or,
    \href{https://docs.julialang.org/en/v1/manual/methods/#Function-like-objects-1}{function-like objects}.\\

    Evaluation:
    {\tiny
    \begin{algorithmblock}
        \begin{juliaverbatim}
eos = EnergyEquation(BirchMurnaghan3rd(224u"bohr^3", 9u"GPa", 4, -4400.3221128178u"eV"))
eos(29.6u"angstrom^3")  # -4399.983486835037u"eV"
        \end{juliaverbatim}
    \end{algorithmblock}
    }
    Notice how we do not need to convert between units! Thank god!\\

    Fitting equations of state:
    {\tiny
    \begin{algorithmblock}
        \begin{juliaverbatim}
nonlinfit(EnergyEquation(BirchMurnaghan3rd(40, 0.5, 4)), volumes, energies)
nonlinfit(EnergyEquation(Murnaghan1st(41, 0.5, 4)), volumes, energies)
nonlinfit(EnergyEquation(Vinet(41, 0.5, 4)), volumes, energies)
        \end{juliaverbatim}
    \end{algorithmblock}
    }
\end{frame}
\endgroup
