\subsubsection{Parsers for \ab{} software I/O}

\begin{frame}[allowframebreaks]{\subsubsecname}
    Most \ab{} software are written in Fortran and has plain text files as input \& output.
    Therefore, we need parsers for both of them. But it is tricky...

        {\footnotesize
            \begin{itemize}
                \item For example, \qe{}'s input adopts the \texttt{Namelist} data structure from
                      Fortran. Julia does not have a parser for Fortran...
                \item A Python package
                      (\href{https://github.com/marshallward/f90nml}{\texttt{f90nml}}) can do that.
                      I wrote a very preliminary package
                      \href{https://github.com/singularitti/PyFortran90Namelists.jl}{\texttt{PyFortran90Namelists.jl}}
                      to call that Python code.
                \item However, it uses \texttt{PyCall.jl}, so sometimes people have trouble installing
                      it if they already have Python installed
                      (see \href{https://github.com/JuliaPy/PyCall.jl\#specifying-the-python-version}{``Specifying the Python version''}).
            \end{itemize}
        }

    For output files, it is even more complicated since they usually do not have a standard
    format, therefore lots of regular expressions need to be used.
    It is extremely error-prone. I used
    \href{https://github.com/jkrumbiegel/ReadableRegex.jl}{\texttt{ReadableRegex.jl}} to build them.
\end{frame}
