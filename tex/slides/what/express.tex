\subsection{What is needed in an \ab{} workflow system?}

\begin{frame}{What is needed in an \ab{} workflow system?}
      They require at least three components:

      \begin{itemize}
            \item Sciency \& technical stuff: calculations, input generation, running external \ab{} software, output analysis...
            \item Dispatcher or job scheduler: wrap all steps in jobs and manage them
            \item User interface: interact with users
      \end{itemize}
\end{frame}

\subsection{What is \express{}?}

\begin{frame}{\subsecname}
      \begin{definitionblock}{\express{}}
            A workflow framework consisting of several extensible, lightweight, high-throughput,
            high-level Julia packages that aims to automate \ab{} calculations for the materials
            science community.
      \end{definitionblock}

      We will mention \texttt{Express.jl} later, but beware that it differs from the
      project name \express{}. We sometimes say \texttt{Express.jl} and \texttt{Express}
      interchangeably.

      \begin{definitionblock}{\texttt{Express.jl}}
            The core package of the \express{} project, which manages and dispatches the rest
            packages in \express{}.
      \end{definitionblock}
\end{frame}

\begin{frame}[allowframebreaks]{What does \express{} include?}
      \begin{figure}[H]
            \centering
            \makebox[\textwidth][c]{\includegraphics[width=0.9\pagewidth]{components}} % See https://tex.stackexchange.com/a/16584/61591
            \caption{Main components of the \express{} project in terms of Julia packages.}
            \label{fig:components}
      \end{figure}

      \framebreak

      \express{} packages can be separated into two kinds: software-neutral and software-specific.
      For the first kind, it includes
            {\footnotesize
                  \begin{itemize}
                        \item \href{https://github.com/MineralsCloud/Express.jl}{\texttt{Express.jl}}
                              provides a high-level interface to all the
                              workflows: file I/O, job scheduling, data analysis, etc.
                              To work with specific software, install the corresponding plugin.
                        \item \href{https://github.com/MineralsCloud/ExpressCommands.jl}{\texttt{ExpressCommands.jl}}
                              is a CLI of \texttt{Express.jl}. It installs an executable
                              `\texttt{xps}' which executes code from configuration files.
                        \item \href{https://github.com/MineralsCloud/EquationsOfStateOfSolids.jl}{\texttt{EquationsOfStateOfSolids.jl}}
                              fits $E(V)$ data to equations of state.
                        \item \href{https://github.com/MineralsCloud/Crystallography.jl}{\texttt{Crystallography.jl}}
                              calculates cell volumes from lattice constants, finds symmetry
                              operations and generates high symmetry points in the Brillouin zone, etc.
                        \item \href{https://github.com/MineralsCloud/PyQHA.jl}{\texttt{PyQHA.jl}}
                              is a Julia wrapper of
                              the Python \texttt{qha} package, which can calculate
                              several thermodynamic properties of both single- and multi-configuration
                              crystalline materials in the framework of quasi-harmonic approximation.
                        \item \href{https://github.com/MineralsCloud/Pseudopotentials.jl}{\texttt{Pseudopotentials.jl}} presents
                              a database for storing and querying pseudopotentials used in \ab{} calculations.
                        \item \href{https://github.com/MineralsCloud/SimpleWorkflows.jl}{\texttt{SimpleWorkflows.jl}}
                              is the skeleton of the workflow system, which
                              defines building blocks, composition rules, and operation order of workflows.
                  \end{itemize}
            }

      For the second kind, since we currently only support \qe{}, it now includes:
      {\footnotesize
      \begin{itemize}
            \item \href{https://github.com/MineralsCloud/QuantumESPRESSOBase.jl}{\texttt{QuantumESPRESSOBase.jl}}
                  declares basic data types and methods
                  for manipulating crystal structures, generating input files for \qe{},
                  error checking before running, etc.
            \item \href{https://github.com/MineralsCloud/QuantumESPRESSOParser.jl}{\texttt{QuantumESPRESSOParser.jl}}
                  parses the I/O files of \qe{} to extract and analyze data.
            \item \href{https://github.com/MineralsCloud/QuantumESPRESSOFormatter.jl}{\texttt{QuantumESPRESSOFormatter.jl}}
                  formats the input files of \qe{}.
            \item \href{https://github.com/MineralsCloud/QuantumESPRESSOCommands.jl}{\texttt{QuantumESPRESSOCommands.jl}}
                  is a CLI that exports the commands of \qe{} in a configurable way.
            \item \href{https://github.com/MineralsCloud/QuantumESPRESSO.jl}{\texttt{QuantumESPRESSO.jl}}
                  is simply a wrapper of the types, methods, and commands defined in
                  the four packages mentioned above.
            \item \href{https://github.com/MineralsCloud/QuantumESPRESSOExpress.jl}{\texttt{QuantumESPRESSOExpress.jl}}
                  is a plugin of \texttt{Express.jl} for handling \ab{} software such as \qe{}.
      \end{itemize}
      }
\end{frame}
