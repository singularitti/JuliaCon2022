\subsection{User interface}

\begin{frame}{Motivation}

    \begin{block}{What should be the most commonly-used interface}
        Since most tasks in different \ab{} calculations are routine, with only
        input parameters changed. It is desirable to make an interface that takes a few
        parameters each time.

        A command line interface plus a template file for the \ab{} software for the sciency
        fixed settings and a configuration file for the computational variables is a
        balanced choice.
    \end{block}

    \begin{block}{Command line interface}
        We want a command line interface to have

        \begin{itemize}
            \item A single topmost command (\texttt{xps})
            \item A few second-tier commands corresponding to different workflows
            \item A few third-tier commands corresponding to different tasks in a workflow
            \item A few arguments, flags, and options for each command mentioned above
        \end{itemize}
    \end{block}

\end{frame}
