\begin{frame}{Run commands with configuration files}

    \begin{columns}[t]
        \begin{column}{0.4\textwidth}
            As stated in previous slides, we could put computational settings
            in a configuration file to change variables dynamically.

            We make use of
            \href{https://github.com/Roger-luo/Configurations.jl}{\texttt{Configurations.jl}},
            which can read TOML files into a Julia \texttt{struct}
            (similar to \href{https://github.com/quinnj/JSON3.jl}{\texttt{JSON3.jl}}).
            We accept JSON and YAML formats as well.

            A typical configuration file looks like this (items may change in the future versions):
        \end{column}

        \begin{column}{0.6\textwidth}
            \begin{minted}[frame=single, linenos]{toml}
            recipe = "eos"
            template = "template.in"
            [cli.mpi]
            np = 128
            [save]
            status = "status.jls"
            eos = "eos.jls"
            [trial_eos]
            type = "bm3"
            values = ["300.44 bohr^3", "74.88 GPa", 4.82]
            [fixed.pressures]
            unit = "GPa"
            values = [-5, -2, 0, 5, 10, 15, 17, 20]
        \end{minted}
        \end{column}
    \end{columns}

\end{frame}
