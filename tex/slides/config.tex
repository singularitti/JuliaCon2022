\begin{frame}[fragile]
    \frametitle{\subsecname}
    \framesubtitle{Run commands with configuration files}

    \begin{columns}[t]
        \begin{column}[T, onlytextwidth]{0.4\textwidth}
            As stated in previous slides, we could put computational settings
            in a configuration file to change variables dynamically.\\

            We use
            \href{https://github.com/Roger-luo/Configurations.jl}{\texttt{Configurations.jl}},
            which can read TOML files into a Julia \texttt{struct}
            (similar to \href{https://github.com/quinnj/JSON3.jl}{\texttt{JSON3.jl}}).
            We accept JSON and YAML formats as well.
        \end{column}

        \begin{column}[T]{0.6\textwidth}
            {\footnotesize A typical configuration file looks like this (items may change in future versions):}
            % \begin{minted}[frame=single, linenos]{toml}
            {\scriptsize
                \begin{algorithmblock}
                    \begin{juliaverbatim}
recipe = "eos"
template = "template.in"
[cli.mpi]
np = 128
[save]
status = "status.jld2"
eos = "eos.jld2"
[trial_eos]
type = "bm3"
values = ["300.44 bohr^3", "74.88 GPa", 4.82]
[fixed.pressures]
unit = "GPa"
values = [-5, -2, 0, 5, 10, 15, 17, 20]
    \end{juliaverbatim}
                \end{algorithmblock}
            }
        \end{column}
    \end{columns}

\end{frame}
