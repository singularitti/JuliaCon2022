\section{Predecessors}

\begin{frame}{VLab}
    \begin{columns}[t]
        \begin{column}{0.4\textwidth}
            \begin{itemize}
                \item The Virtual Laboratory for Earth and Planetary Materials (VLab), was
                      funded by the NSF in 2004 at the Minnesota Supercomputing Institute.
                \item VLab is a cyberinfrastructure consisting a fully integrated web
                      portal, web services, and databases for ab initio calculations of
                      planetary materials.
            \end{itemize}
        \end{column}

        \begin{column}{0.6\textwidth}
            \begin{tikzpicture}[baseline=(current bounding box.north)]
                \draw (0, 0) node {\includegraphics[width=\textwidth]{vlab}};
                \draw (1, -2.5) node {\url{http://mineralscloud.com/resources/}};
            \end{tikzpicture}
        \end{column}
    \end{columns}
    \footcitetext{DASILVA2007321}

    \begin{tikzpicture}[overlay, remember picture]
        \node[xshift=-1cm,yshift=-1cm] at (current page.north east) {\qrcode[height=2cm]{http://mineralscloud.com/resources/}};
    \end{tikzpicture}
\end{frame}

\begin{frame}{\texttt{qha}}
    \begin{columns}[t]
        \begin{column}{0.4\textwidth}
            We drew inspiration from our published and unpublished code, e.g. \texttt{qha},
            when developing the current workflows.

            For example, \texttt{qha} runs with a human-readable
            configuration file and a given input.
            The same mechanism is adopted in \express{}.
            See subsection \ref{ssec:ui}.
        \end{column}

        \begin{column}{0.6\textwidth}
            \begin{tikzpicture}[baseline=(current bounding box.north)]
                \draw (0, 0) node {\includegraphics[width=\textwidth]{qha}};
                \draw (1, -3) node {\url{https://github.com/MineralsCloud/qha}};
            \end{tikzpicture}
        \end{column}
    \end{columns}

    \begin{tikzpicture}[overlay, remember picture]
        \node[xshift=-1cm,yshift=-1cm] at (current page.north east) {\qrcode[height=2cm]{https://github.com/MineralsCloud/qha}};
    \end{tikzpicture}
\end{frame}
